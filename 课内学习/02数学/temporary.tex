\text{在各项均为正数的等比数列}\{a_n\}\text{ 中,}a_6=2a_5+3a_4\text{,若存在两项}a_m\mathrm{,}a_k\text{ 使得 }\sqrt{a_ma_k}=3a_1\text{,则}\frac{1}{m}+\frac{4}{k}\text{的最小值为()(016,3)}

$\begin{aligned}&\text{已知椭圆}E\text{ 的离心率为}\frac{\sqrt{2}}{2}\text{,椭圆}E\text{ 上一点 }P\text{ 到左焦点的距离的最小值为 }\sqrt{2}-1.(2)\text{已知直线}l\text{ 与椭圆}E\text{ 交于 }M\mathrm{~、}N\text{ 两点,且}OM\perp ON\text{,求心OMN面积的取值范围(017,9)}.\end{aligned}$

$\text{点}P(3,a)\text{ 关于直线}x+y-a=0\text{ 的对称点在圆}(x-2)^2+(y-4)^2=13\text{ 内,则实数}a\text{ 的取值范围是()(017,7)}$

$\text{已知抛物线}\Gamma:\quad y^{\prime}=4x,\text{ 在}\Gamma\text{ 上有一参}A\text{位于第一参限,设A的纵坐标为}.a(a>0).\text{ 直线}:\quad x=-3\text{ ,抛物线上有一异于点A的动点P,P在直线}\text{上的投影为点H,直线AP与直线的交点为}Q.\text{ 若在P的位置变化过程中,}\left|HQ\right|>4\text{ 恒成立,求a的取值范围(022,18)}.$